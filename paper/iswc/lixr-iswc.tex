\documentclass{llncs}

\usepackage{url}

\begin{document}

\title{LIXR: Quick, succinct conversion of XML to RDF}

\author{
John P. McCrae\textsuperscript{1} \and
Philipp Cimiano\textsuperscript{2}}

\institute{
    \textsuperscript{1}Insight Centre for Data Analytics, National University of Ireland, Galway\\
    \email{john@mccr.ae}\\
    \textsuperscript{2}Cognitive Interaction Technology, Cluster of Excellence,
    Bielefeld University\\
    \email{cimiano@cit-ec.uni-bielefeld.de}
}

\maketitle
\begin{abstract}
This paper presents LIXR, a system for converting between RDF and XML. LIXR is based on   domain-specific language embedded into the Scala programming language.
It supports the definition of transformations of 
datasets from RDF to XML in a declarative fashion, while still maintaining the flexibility of a full
programming language environment. We directly compare this system to other systems
programmed in Java and XSLT and show that the LIXR implementations are significantly
shorter in terms of lines of code, in addition to being conceptually simple to understand.

\end{abstract}

%\category{H.4}{Information Systems Applications}{Miscellaneous}
%\category{D.2.8}{Software Engineering}{Metrics}[complexity measures, performance measures]

%\terms{Theory}

\keywords{RDF, XML, Scala, format conversion}

\section{Introduction}

An important aspect towards realizing a web of data is the conversion of legacy resources into the 
Resource Description Framework~\cite[RDF]{cyganiak2014rdf}. There are tools and
W3C recommendations\footnote{\url{http://www.w3.org/TR/rdb-direct-mapping/}}
supporting the transformation of relational databases into RDF (e.g. D2RQ
\cite{bizer2004d2rq}), and even declaratively languages such as
R2RML\footnote{\url{http://www.w3.org/TR/r2rml/}}. Besides the relational model,
legacy data represented in XML is quite frequent. While there exist generic
mechanisms for transforming XML data into some other form, such as Extensible
Stylesheet Language Transformations (XSLT), these mechanisms are not ideal for
the conversion into RDF for the following reasons: 


\begin{enumerate}
\item The generated RDF typically contains more triples than necessary due to the fact that generic
converters create both a property and a node for each individual element in the XML.
\item It is uncommon for XML documents to reuse URLs from other resources. For
example it is typical for a resource to reuse the data categories of Dublin Core~\cite{weibel1998dublin}, 
but to recast them under a new namespace, that is not compatible with RDF.
\item XML provides no generic mechanisms for the representation of external
    links by URIs, using a proprietary linking schema instead such as
    XLink~\cite{derose2001xml}. 
\end{enumerate}

%These points can be further examined by considering a mini-example from
%the META-SHARE data described further below.
%
%\begin{verbatim}
%<languageInfo>
%  <languageId>bg</languageId>
%</languageInfo>
%\end{verbatim}
%
%In this case, it is unlikely that we would wish to create two properties for the 
%elements ({\tt 
%languageInfo}, {\tt languageId}) and an intermediate node, but instead
%would rather re-use a single existing property from Dublin Core ({\tt
%dc:language}) and even map the language identifier to a common repository such
%as LexVo~\cite{de2008language}. Such choices cannot be made with a 
%generic converter.

Further, transformation from XML to some other format are typically specified by means of  
XSLT, an extension of XSLT such as
Krextor~\cite{lange2009krextor}, a specific mapping language such as
RML~\cite{dimou2014rml} or 
by writing a short script in some programming language. Thus, these transformations
are generally very verbose, as they must repeat many standard RDF modelling
structures, and unidirectional, as they are not well-formulated to cope with the 
polymorphic nature of RDF.

In order to meet these shortcomings, we developed a new system for specifying the
translation of XML documents into RDF and \emph{vica versa}, which we call the Lightweight Invertible
XML and RDF language (LIXR, pronounced `elixir').
%This system builds on a 
%Scala domain-specific language~\cite[DSL]{fowler2010domain,wampler2008programming}, which allows the conversion to be
LIXR is significantly more compact than existing systems and 
allows for transformation in both the direction of RDF to XML and from XML
to RDF.

%We describe the system and the features of the domain-specific language domain in Section
%\ref{sec:dsl}. Then we will consider a detailed use case, where we used the LIXR
%system to convert a very complicated XML schema into RDF, namely for the META-SHARE
%resources~\cite{mccrae2015ontology,piperidis2012meta}, in Section \ref{sec:metashare}. Next, we will present some qualitative evaluation
%of the methodology to demonstrate its brevity and discuss these results in Section
%\ref{sec:results} and conclude in Section \ref{sec:conclusion}.

\section{The LIXR Language}
\label{sec:dsl}

The LIXR language was created as a domain-specific language based on the Scala
Language. This choice was made as Scala has an exceptional amount of freedom in
expression, allowing us to compactly and clearly state transformations, although
there would be some learning curve for those not familiar with the Scala
language.
%in a manner
%that would be more cumbersome in other languages such as a fully RDF
%approach~\cite{dimou2014rml}. 
%Moreover, the fact that Scala
%compiles to Java bytecode allows us to reuse existing libraries for handling
%RDF, including the Jena library\footnote{\url{http://jena.apache.org/}}.

The basic structure of LIXR is inspired by XSLT and is based around \emph{handlers},
which describe the action that should be taken when a specific XML element is 
encountered. These handlers are stated by linking an XML element name to a list
of \emph{generators} with the {\tt -->} operator. For example the following LIXR
expression can be used:

\begin{verbatim}
xml.language --> (
  dc.language > content
)
\end{verbatim}

This associates the XML element {\tt <xml:language>} to
generating the triple \emph{s} {\tt dc:language} {\tt "}\emph{c}{\tt "}, 
where \emph{s} is the current subject node, and \emph{c} is the text content of
the node\footnote{Note that for technical reasons the {\tt .} is used to conjoin the namespace
to the local name instead of the customary {\tt :}. This, in fact, is 
a dynamic call, a relatively recent feature of the Scala language.} More typically
the converter works by means of two features: firstly, {\tt node}s instruct the RDF generation
to create a new node in the RDF graph and use it as subject for all triples
from this point in the generation. Secondly, the {\tt handle} tells the XML parser to look for all 
children matching a given element and call the appropriate handler for each matching
case. For example:

\begin{verbatim}
xml.metadata --> (
  node("http://.../metadata")(
    handle(xml.language),
    handle(xml.source)
  )
)
\end{verbatim}

This code asks the RDF generator to create a new root node with the given URI. The parser
then looks for matching children and calls the appropriate handlers
(such as in the first example). 
%In the opposite direction (RDF to XML), it is
%possible to generate the XML by walking the RDF diagram as a tree. 
More detail
of the language can be found on
Github\footnote{\url{https://github.com/liderproject/lixr}}.
%\footnote{
%The LIXR language allows for cycles in RDF graphs only by double naming of
%{\tt node}s} generating a
%element for any handler where all of its mandatory generators and at least one 
%generator is active. For example, if a triple is generated this is considered
%mandatory, however {\tt handle}s are not mandatory. In some cases, this means that
%`round-tripping', that is the conversion of RDF to XML and back to RDF, may not
%be possible. For example, the {\tt substring} function can not be used in the
%inverse direction and is defined by means of two lambda functions, based on
%the generic {\tt transform} generator:
%
%\begin{verbatim}
%def substring(tg : TextGenerator, s : Int, e : Int) =
%  transform(tg)
%           (string => string.substring(s, e))
%           (throw new UnsupportedOperationException())
%\end{verbatim}
%
%This defines a substring as a {\tt transform} of a text element, {\tt tg}, which
%is realized in the XML-to-RDF direction by the lambda function that calls the 
%Scala built-in function {\tt substring} and in the RDF-to-XML direction by
%a lambda that raises an exception.
%
%A truly bidirectional operator, for example to remove a prefix from the XML when
%generating RDF can be introduced by a custom converter.
%We can define a transform function as follows
%
%\begin{verbatim}
%def removePrefix(tg : TextGenerator) =
%  transform(tg)
%           (string => string.drop(10))
%           (string => "myprefix__" + string) 
%\end{verbatim}

%In addition, there are a number of other features provided by the LIXR langauge:
%
%\begin{itemize}
%\item `Backlink' triple generators where the current node is the object of the triple
%\item Features for generating unique fragment identifiers and for generating globally unique 
%identifiers (UUIDs).
%\item Conditional statements ({\tt when... or... otherwise}), which function as
%{\tt if... then... else}, but do not clash with the Scala keywords. In inverse mode
%these function on a first success basis.
%\item Iterators {\tt forall}, allowing for direct iteration. Handlers are the 
%preferred from of traversing the XML document, but an iterator may be more appropriate 
%if there is a element the transformation of which is dependent on where it appears in the
%XML graph.
%\item Variables may be assigned, which can be used later in the translation.
%\end{itemize}

%\section{Converting META-SHARE to RDF}
%\label{sec:metashare}
\begin{table*}[h]
\begin{center}
\begin{tabular}{p{4cm}|cccc}
Name & Tags & Implementation & LoC & LoC/Tag \\
\hline
TBX & 48 & Java & 2,752 & 57.33 \\
CMDI & 79 & XSLT & 404 & 5.11 \\
CMDI & 79 & XSLT (No closing tags) & 255 & 3.22 \\
CMDI & 79 & XSLT (Using Krextor~\cite{lange2009krextor}) & 454 & 5.75 \\
CMDI & 79 & RML~\cite{dimou2014rml} & 339 & 4.29 \\
\hline
CMDI & 79 & LIXR & 176 & 2.23 \\
TBX & 48 & LIXR & 197 & 4.10 \\
MetaShare & 730 & LIXR & 2,487 & 3.41 \\
\end{tabular}
\end{center}
\caption{\label{tab:locs}Comparison of XML to RDF mapping implementations,
by number of elements in XML schema, and non-trivial lines of code (LoC)}
\vspace{-10mm}
\end{table*}

%META-SHARE is a part of the META-NET project for the management of language resources. 
%META-SHARE attempts to catalogue and provide detailed metadata about the whole
%lifecycle of the resource. A META-SHARE record of a language resource
%contains not only resource specific information, such as the language, type
%or annotation scheme of the resource, but also information concerning the
%creation, intended use, associated publications and funding projects. This 
%information may contain very specific details, such as the addresses of the individual
%institute who created the resource.
%
%The META-SHARE schema itself is quite complex, consisting of 111 complex types
%and 207 simple types.
%Writing a\textsc{} converter to RDF is thus not a trivial exercise. Instead of writing an
%XSL transformation, we used LIXR to express the conversion from XML to RDF in a declarative fashion.
%An extract of LIXR applied to the case of converting META-SHARE data to RDF is given in what follows:
%
%{\scriptsize
%\begin{verbatim}
%object Metashare extends eu.liderproject.lixr.Model {
% val dc = Namespace("http://purl.org/dc/elements/1.1/")
% val ms = Namespace("http://purl.org/ms-lod/MetaShare.ttl#")
% val msxml = Namespace("http://www.ilsp.gr/META-XMLSchema")
%
% msxml.resourceInfo --> (
%   a > ms.ResourceInfo,
%   handle(msxml.identificationInfo)
% )
%
% msxml.identificationInfo --> (
%   forall(msxml.resourceName)(
%     dc.title > (content @@ att("lang"))
%   )
% )
%}
%\end{verbatim}}
%In this example, we first create our model extending the basic LIXR model and
%define namespaces as dynamic Scala objects. We then make two mapping
%declarations for the elements {\tt resourceInfo} and {\tt identifi\-cationInfo}. LIXR (as
%XSLT) simply searches for a matching declaration at the root of the XML document
%to begin the transformation. Having matched the {\tt resourceInfo} element, the system
%first generates the triple that states that the base element has type
%{\tt ms:resourceInfo}, and then `handles' any children {\tt
%identi\-ficationInfo} element by
%searching for an appropriate rule for each one. For {\tt identificationInfo} the
%system generates a triple using the {\tt dc:title} property whose value is the
%content of the {\tt resourceName} element tagged with the language given by the
%attribute {\tt lang}.


\section{Evaluation}
\label{sec:results}


To evaluate the effectiveness of our approach we compared directly with four other
XML to RDF transformations in terms of an objective measure, that is \emph{lines of code}. The other transformation programs were written by the lead author of the
project\footnote{The lead author has over 5 years experience in all languages}, and
reimplemented using the LIXR language.
In particular, \emph{lines of code} is easily measured and it
has been claimed~\cite{mcconnell2004code} that the average number of errors made per lines of code is 
approximately constant for a given programmer, regardless of what language he or
she is programming in. As such, lines of code can be a good proxy not only for
ease of development but also for software quality. As such, we measure the
code in terms of \emph{non-trivial} lines of code, where a line of code
is considered trivial if it only contains closing brackets or braces or is empty. 

We consider three existing XML schemas as targets to be converted into RDF: the TermBase eXchange format (TBX, ISO
30042:2008), the
META-SHARE schema\footnote{\url{http://metashare.ilsp.gr/META-XMLSchema/v3.0/}} 
for representing very rich metadata about language resource, and the Component Metadata Initiative
(CMDI) used by
CLARIN\footnote{\url{http://catalog.clarin.eu/ds/ComponentRegistry/rest/registry/profiles/clarin.eu:cr1:p\_1288172614026/xsd}}.

%The first data source we considered was the
%TermBase eXchange format~\cite[TBX]{iso30042}, a format for representing multilingual
%terminology, which was an ISO standard and has seen wide adoption in both industry
%and academia. This model was first written as a Java application, and was significantly
%complex due to the interface with the XML and RDF libraries. In contrast, the LIXR 
%implementation was many times shorter as LIXR provided built in mechanisms for 
%handling both RDF and XML. 
%
%
%Secondly, we looked at the CLARIN XML schema also known as
%Component Metadata Initiative~\cite{broeder2012cmdi}, which is a format similar
%to META-SHARE for representing metadata about language resources. As described
%in section \ref{sec:metashare}, this is a complex schema of a large number of 
%elements that required significant mapping.
%The CMDI model
%has a small common header and then a second schema that is specific to the
%institute publishing the metadata. In particular, we used the profile for 
%Dublin Core\footnote{\url{http://catalog.clarin.eu/ds/ComponentRegistry/rest/registry/profiles/clarin.eu:cr1:p\_1288172614026/xsd}},
%which is shared across a large number of metadata publishers. 
%This conversion
%was first implemented in XSLT, and in fact using a non-standard XML serialization
%called Jade\footnote{\url{http://jade-lang.com/}}, which uses tab-based bracketing 
%(like the Python language) and thus does not have closing elements like standard XML.
%As such this compact syntax does reduce the number of lines-of-code, however we 
%still see that the XSLT formalism is significantly more verbose than LIXR. All
%the results are presented in Table \ref{tab:locs}. 

The results of the comparison in terms of lines of code for the datasets and the different transformations is given in Table \ref{tab:locs}.
The results show that 
LIXR leads to significantly shorter code in terms of lines-of-code than the other methods
we attempted. In fact, we observe a 10-fold reduction of effort over directly
writing a converter in a general purpose programming language (Java) and we see
a halving of effort in comparison to using a specialist language (XSLT, Krextor
or RML). In
addition, we note that the reduction is such that only a few lines of code are
needed for each element class. We note that all of the systems ran quickly over
the data we tested and thus we do not believe that processing time or memory are
radically different between these implementations.
%This is in addition to the obvious benefits of invertibility and
%the fact that LIXR is likely more correct as the formalism is more concise and 
%natural for XML to RDF conversion. 

\section{Conclusion}
\label{sec:conclusion}

We have presented a declarative yet flexible approach supporting the conversation of XML to RDF. The approach
is based on a domain-specific language embedded in the Scala programming language. We have shown that this converter supports the implementation of concise and shorter conversion programs than with other transformation languages.

In addition to the reduction in effort using this approach, we also note several
other advantages of the LIXR approach that could easily be added, 
due to its declarative nature,
including stream processing of XML, reverse mapping from RDF to XML and extraction
of an ontology from the mapping.
%\begin{itemize}
%\item We can easily switch to using a stream-based parse for XML (e.g., SAX)
%so we can process large files without having to use as much memory. LIXR
%contains only a few elements that are not suitable for stream processing, but
%are in this case handled by means of a cached state in the generation. As such,
%LIXR manages to be relative expressive but still allows for high performance.
%\item A reverse mapping can be extracted that re-generates the XML from the
%outputted RDF.
%\item We can extract the type, range and domain of RDF entities generated
%during this procedure. This allows an ontology to be easily extracted with
%domains and ranges of properties and classes based on the export procedure.
%\end{itemize}


\section*{Acknowledgments}

%This work has been funded by the LIDER project funded under the European Commission 
%Seventh Framework
%Program (FP7-610782), 
%the Cluster of Excellence Cognitive
%Interaction Technology `CITEC' (EXC 277) at Bielefeld University, which
%is funded by the German Research Foundation (DFG), and the Insight Centre for
%Data Analytics which
%is funded by the Science Foundation Ireland under Grant Number SFI/12/RC/2289.

\bibliographystyle{splncs03}
\bibliography{lixr-iswc}

\end{document}
