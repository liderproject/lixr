\documentclass{acm_proc_article-sp}

\begin{document}

\title{LIXR: Fast, efficient conversion of XML to RDF and back again}

\numberofauthors{1}

\author{
\alignauthor
John P. McCrae\\
       \affaddr{Cognitive Interaction Technology, Cluster of Excellence}\\
       \affaddr{Bielefeld University}\\
       \affaddr{Bielefeld, Germany}\\
       \email{john@mccr.ae}
}

\maketitle
\begin{abstract}
\end{abstract}

\category{H.4}{Information Systems Applications}{Miscellaneous}
\category{D.2.8}{Software Engineering}{Metrics}[complexity measures, performance measures]

\terms{Theory}

\keywords{ACM proceedings, \LaTeX, text tagging} % NOT required for Proceedings

\section{Introduction}
\subsection{Mapping META-SHARE to RDF [JPM]}

\label{sec:conversion}

When translating XML documents into RDF, one of the most common approaches is
based on exploiting Extensible Stylesheet Language Transformations
(XSLT)~\cite{wustner2002converting,van2008xml,borin2014representing}. However, XSLT has a number of disadvantages
for this task, in particular as it has limited extensibility and cannot be
applied to streaming data or as an invertible transform.

The META-SHARE schema itself is quite complex, consisting of 111 complex types
and 207 simple types.
Writing a converter to RDF is thus not a trivial exercise. Instead of writing an
XSL transformation, we opted to follow a different approach by developing a
\emph{domain-specific language} (DSL) ~\cite{fowler2010domain} that allows us to express the conversion from XML to RDF in a declarative fashion.
This language is called the \textit{Lightweight Invertible XML to RDF Mapping
Language} (LIXR, pronounced `elixir') and a simple example of a
LIXR mapping is given below:
{\footnotesize
\begin{verbatim}
object Metashare extends eu.liderproject.lixr.Model {
 val dc = Namespace("http://purl.org/dc/elements/1.1/")
 val ms = Namespace("http://purl.org/ms-lod/MetaShare.ttl#")
 val msxml = Namespace("http://www.ilsp.gr/META-XMLSchema")

 msxml.resourceInfo --> (
 a > ms.ResourceInfo,
 handle(msxml.identificationInfo)
 )

 msxml.identificationInfo --> (
 forall(msxml.resourceName)(
 dc.title > (content @@ att("lang"))
 )
 )
}
\end{verbatim}}
In this example, we first create our model extending the basic LIXR model and
define namespaces as dynamic Scala objects\footnote{This is a newer feature of
Scala only supported since 2.10 (Jan 2013)}. We then make two mapping
declarations for the tags {\tt resourceInfo} and {\tt identificationInfo}. LIXR (as
XSLT) simply searches for a matching declaration at the root of the XML document
to begin the transformation. Having matched the {\tt resourceInfo} tag, the system
first generates the triple that states that the base element has type
{\tt ms:resourceInfo}, and then `handles' any children {\tt identificationInfo} tags by
searching for an appropriate rule for each one. For {\tt identificationInfo} the
system generates a triple using the {\tt dc:title} property whose value is the
content of the {\tt resourceName} tag tagged with the language given by the
attribute {\tt lang}.
\begin{table}
\begin{center}
\begin{tabular}{p{4cm}|cccc}
Name & Tags & Implementation & LoC & LoC/Tag \\
\hline
TBX & 48 & Java & 2,752 & 57.33 \\
CLARIN (OLAC-DMCI) & 79 & XSLT & 404 & 5.11 \\
CLARIN (OLAC-DMCI) & 79 & XSLT (Compact Syntax) & 255 & 3.22 \\
\hline
TBX & 48 & LIXR & 197 & 4.10 \\
CLARIN (OLAC-DMCI) & 79 & LIXR & 176 & 2.23 \\
MetaShare & 730 & LIXR & 2,487 & 3.41 \\
\end{tabular}
\end{center}
\caption{\label{tab:locs}Comparison of XML to RDF mapping implementations,
by number of tags in XML schema, and non-trivial lines of code (LoC)}
\end{table}
To evaluate the effectiveness of our approach we compared directly with two other
XML to RDF transformations, we had carried out in this project, and
reimplemented them using the LIXR language. In particular these were the TBX
model~\cite{iso30042} as well as the OLCA-DMCI profile of the CLARIN
metadata~\footnote{\url{http://catalog.clarin.eu/ds/ComponentRegistry/rest/registry/profiles/clarin.eu:cr1:p\_1288172614026/xsd}}. In table \ref{tab:locs}, we see the
effort to implement these using LIXR is approximately half of using XSLT and
about ten times less than writing a converter from scratch.
In addition to the reduction in effort using this approach, we also note several
other advantages of the LIXR approach, due to its declarative declaration
\begin{itemize}
\item We can easily switch to using a stream-based parse for XML (e.g., SAX)
so we can process large files without having to use much memory
\item A reverse mapping can be extracted that re-generates the XML from the
outputted RDF
\item We can extract the type, range and domain of RDF entities generated
during this procedure. This export formed the initial version of the
ontology described in this paper
\end{itemize}

\section{Acknowledgments}

\bibliographystyle{abbrv}
\bibliography{lixr-semantics}

\end{document}
